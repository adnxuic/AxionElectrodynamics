%===================================
% note-setup-leftsidebox.tex
% huanengchen@foxmail.com 2025-08-12
%===================================
% 参考:https://tex.stackexchange.com/questions/59702/suggest-a-nice-font-family-for-my-basic-latex-template-text-and-math

%===================================
% 页面和间距
%===================================

\usepackage[a4paper, margin=1in]{geometry} % 具体设置参考 geometry 宏包
\setlength{\parindent}{0pt} % 取消首行缩进
\usepackage{parskip} % 形成段落间的间距
\linespread{1.25} % 修改行距

%===================================
% 字体
%===================================

\usepackage{fix-cm} % 允许 Computer Modern 字体按需缩放,避免缺失字号警告

%===================================
% 编辑体验
%===================================

\usepackage{float} % 优化浮动体
\usepackage[shortlabels,inline]{enumitem} % 优化列表
\usepackage{appendix} % 优化附录
\usepackage{etoolbox} % 提供命令补丁功能

%===================================
% 表格
%===================================

\usepackage{booktabs, multirow, multicol}
\usepackage{tabularx} % 表格自动换行,调整表格宽度 
\usepackage{makecell} % 单元格内换行
\usepackage{threeparttable} % 给表格添加脚注,参考:https://tex.stackexchange.com/questions/6090/clickable-table-footnote
\usepackage{ltablex} % 跨页表格

%===================================
% 参考文献
%===================================

\usepackage[numbers,sort&compress]{natbib} % 参考文献样式
\bibliographystyle{ieeetr} % 顺序编码制

%===================================
% 颜色
%===================================

\usepackage[dvipsnames, x11names, table]{xcolor} % 参考:https://tex.stackexchange.com/questions/659036/option-selecting-named-colours-provided-by-the-xcolor-package

%===================================
% 支持插入图片及子图
%===================================

\usepackage{graphicx}
\graphicspath{
    {./figure/}{./figures/}{./image/}{./images/}{./graphic/}{./graphics/}{./picture/}{./pictures/}
} % 用于存放图片的目录,这样引用图片的时候就不需要指定目录
\usepackage{subcaption}

%===================================
% 算法和伪代码
%===================================

\usepackage[linesnumbered, ruled, longend, lined]{algorithm2e} % 参考 algorithm2e 宏包文档
\DontPrintSemicolon % 不打印分号
\setlength{\algomargin}{2em} % 设置算法缩进使得行号在线框内
\renewcommand{\CommentSty}[1]{\normalsize\textit{#1}} % 设置注释的字体样式为意大利斜体,字体大小为 \normalsize

%===================================
% 代码
%===================================

% \usepackage{minted} % 参考 minted 宏包文档

% % 代码行号样式
% \renewcommand{\theFancyVerbLine}{
% \sffamily
% \textcolor{gray}{
% \footnotesize\oldstylenums{
% \arabic{FancyVerbLine}}}}

% % 行间代码环境
% \setminted{
%     style=colorful, % 设置代码风格,可选的代码风格参考:https://pygments.org/styles/
%     numbers=left, % 显示行号
%     numbersep=2pt, % 行号与代码的距离
%     mathescape, % 允许在代码注释中使用数学公式
%     breaklines, % 允许代码自动断行 
%     fontsize=\footnotesize, % 设置代码字体大小
%     frame=single, % 设置代码框
%     framerule=0.5pt, % 设置代码框线宽
%     resetmargins, % 重置代码边距
% }

% % 行内代码环境
% \setmintedinline{
%     style=colorful, % 设置代码风格,可选的代码风格参考:https://pygments.org/styles/
%     fontsize=\footnotesize, % 设置代码字体大小
%     breakbytokenanywhere, % 允许行内代码在任意位置断行
%     breaklines, % 允许行内代码自动断行
% }

%===================================
% 超链接
%===================================

\usepackage{hyperref}
\hypersetup{
    bookmarksopen=true, % 启用书签
    colorlinks=true, % 启用颜色
    linkcolor=red, % 内部链接的颜色
    linktoc=all, % 设置目录中的页码和标题都能够跳转
    citecolor=violet, % 引用链接的颜色
    urlcolor=magenta, % 外部链接的颜色
}

% 自定义 autoref 的引用格式
% \def\figureautorefname{Figure} % 将 "Figure" 改为 "图"
% \def\tableautorefname{Table}  % 将 "Table" 改为 "表"
% \def\equationautorefname{Equation} % 将 "Equation" 改为 "公式"

% 内联错误着色命令,用于高亮手稿中的错误片段
\newcommand{\error}[1]{\textcolor{red}{#1}}

\usepackage{csquotes}

%===================================
% 自定义命令
%===================================

% 彩色边框盒子,默认红色边框,内容为黑色
\newcommand{\colorboxed}[2][red]{%
    \begingroup
    \color{#1}\boxed{\color{black}#2}%
    \endgroup
}

%===================================
% 数学公式
%===================================

\usepackage{amsmath, amsthm, amsfonts, amssymb} % 用于加载数学公式、花体字母和数学字符
\usepackage{mathtools}
\usepackage[scr=boondoxo]{mathalfa} % 提供支持小写的 \mathscr 字体,避免 rsfs10 缺字
\usepackage{bm}
\usepackage{extarrows}
\usepackage[noabbrev]{cleveref} % 多公式引用,必须放在 hyperref 宏包的后面,参考:https://tex.stackexchange.com/questions/314217/how-i-can-refer-multiple-equation-in-latex

\allowdisplaybreaks[1] % 多行公式换页,1 为尽量避免换页
\crefname{equation}{}{} % 设置非首字母大写的引用格式
\Crefname{equation}{}{} % 设置首字母大写的引用格式
\crefrangeformat{equation}{(#3#1#4)-(#5#2#6)} % 多公式引用的格式

% --- Custom Definitions (as per user request) ---

% Mathematical constants (roman)
\newcommand{\me}{\mathrm{e}}
\newcommand{\mi}{\mathrm{i}}
\newcommand{\mpi}{\mathrm{\pi}} % Upright pi

% Differential operator (roman)
\newcommand{\md}{\mathrm{d}}

% Vectors (bold)
\newcommand{\vecS}{\bm{\hat{S}}}
\newcommand{\vecsigma}{\bm{\hat{\sigma}}}
\newcommand{\vecn}{\bm{n}}

% Hilbert space
\newcommand{\hilbert}{\mathcal{H}}

% Bra-Ket notation
\newcommand{\ket}[1]{|#1\rangle}
\newcommand{\bra}[1]{\langle#1|}
\newcommand{\braket}[2]{\langle#1|#2\rangle}
\newcommand{\ketbra}[2]{|#1\rangle\langle#2|}

% Hat for operators
\newcommand{\op}[1]{\hat{#1}}
\newcommand{\opI}{\hat{\mathbb{I}}}

% 附录中的公式编号使用 A.1、B.1 等格式
\apptocmd{\appendix}{
    \setcounter{equation}{0}%
    \setcounter{chapter}{0}%
    \renewcommand{\chaptername}{Appendix}%
    \renewcommand{\chapterautorefname}{Appendix}%
    \renewcommand{\thechapter}{\Alph{chapter}}%
    \renewcommand{\thesection}{\thechapter.\arabic{section}}%
    \renewcommand{\thesubsection}{\thesection.\arabic{subsection}}%
    \renewcommand{\thesubsubsection}{\thesubsection.\arabic{subsubsection}}%
    \renewcommand{\theequation}{\thechapter.\arabic{equation}}%
    \renewcommand{\theHchapter}{\Alph{chapter}}%
    \renewcommand{\theHsection}{\theHchapter.\arabic{section}}%
    \renewcommand{\theHsubsection}{\theHsection.\arabic{subsection}}%
    \renewcommand{\theHsubsubsection}{\theHsubsection.\arabic{subsubsection}}%
    \renewcommand{\theHequation}{\theHchapter.\arabic{equation}}%
}{}{}

%===================================
% 边注
%===================================

% 设置边注的字体大小
\let\oldmarginpar\marginpar
\renewcommand{\marginpar}[1]{\oldmarginpar{\footnotesize #1}}

%===================================
% 页脚和页眉
%===================================

\usepackage{fancyhdr} % 参考:https://tex.stackexchange.com/questions/732462/chapter-number-in-the-header-with-chapter/732464?noredirect=1#comment1824660_732464 
\usepackage{lastpage} % 获取总页码
\setlength{\headheight}{15pt} % 设置页眉高度,避免 fancyhdr 警告

% 重新定义 \author 和 \date 命令用于页眉
\makeatletter
\let\oldauthor\author
\renewcommand{\author}[1]{\oldauthor{#1}\def\myauthor{#1}}
\let\olddate\date
\renewcommand{\date}[1]{\olddate{#1}\def\mydate{#1}}
\makeatother

% 详细参数参考 fancyhdr 宏包
\pagestyle{fancy} % 设置文档的页面样式为 fancy,这意味着页眉和页脚将使用 fancyhdr 宏包提供的自定义格式
\fancyhf{} % 清空原本的页脚页眉样式

% 自定义页眉
\fancyhead[L]{\myauthor} % 左侧显示作者
\fancyhead[R]{\mydate} % 右侧显示日期

\cfoot{\thepage\ / \pageref*{LastPage}} % 自定义页脚,参考:https://tex.stackexchange.com/questions/227/how-can-i-add-page-of-on-my-document

%===================================
% 字体
%===================================

% Times New Roman 风格字体设置
\usepackage{newtxtext,newtxmath}

%===================================
% 定理盒子
%===================================

% 修改 proof 环境的引导词为 Proof,样式为加粗无斜体
\renewcommand*{\proofname}{\normalfont\bfseries Proof}

% 导入 thmtools 宏包,使用 \declaretheorem 命令来定义各种定理环境(比 \newtheorem 命令更加方便)
\usepackage{thmtools}

% 定义环境使用的 `\declaretheorem` 命令参数包括:
% - `style`: 定理环境样式,amsthm 内置的样式包括
%   - plain(默认):引导词是正体,内容是斜体
%   - definition:引导词和内容都是正体
%   - remark:引导词是斜体,内容是正体
% - `name`:显示在正文中的引导词(不等于环境的名称)
% - `numbered`:是否开启编号
% - `numberwithin`、`sibling`:定义编号规则,例如:
%   - `numberwithin=section`:基于 section 编号
%   - `sibling=theorem`:共享 `theorem` 环境的编号

% 采用 plain 样式,定义 `theorem`/`theorem*`、`law`/`law*`、`corollary`/`corollary*`、`lemma`/`lemma*`、`claim`/`claim*` 环境

\declaretheorem[style=plain, name=Theorem, numbered=yes, numberwithin=section]{theorem}
\declaretheorem[style=plain, name=Theorem, numbered=no]{theorem*}

\declaretheorem[style=plain, name=Law, numbered=yes, sibling=theorem]{law}
\declaretheorem[style=plain, name=Law, numbered=no]{law*}

\declaretheorem[style=plain, name=Corollary, numbered=yes, sibling=theorem]{corollary}
\declaretheorem[style=plain, name=Corollary, numbered=no]{corollary*}

\declaretheorem[style=plain, name=Lemma, numbered=yes, sibling=theorem]{lemma}
\declaretheorem[style=plain, name=Lemma, numbered=no]{lemma*}

\declaretheorem[style=plain, name=Claim, numbered=yes, sibling=theorem]{claim}
\declaretheorem[style=plain, name=Claim, numbered=no]{claim*}

% 采用 definition 样式,定义 `definition`/`definition*`、`example`/`example*`、`problem`/`problem*` 环境

\declaretheorem[style=definition, name=Definition, numbered=yes, numberwithin=section]{definition}
\declaretheorem[style=definition, name=Definition, numbered=no]{definition*}

\declaretheorem[style=definition, name=Example, numbered=yes, numberwithin=section]{example}
\declaretheorem[style=definition, name=Example, numbered=no]{example*}

\declaretheorem[style=definition, name=Problem, numbered=yes, numberwithin=section]{problem}
\declaretheorem[style=definition, name=Problem, numbered=no]{problem*}

% 采用 remark 样式,定义 `remark`/`remark*`、`note`/`note*` 环境

\declaretheorem[style=remark, name=Remark, numbered=yes, numberwithin=section]{remark}
\declaretheorem[style=remark, name=Remark, numbered=no]{remark*}

\declaretheorem[style=remark, name=Note, numbered=yes, numberwithin=section]{note}
\declaretheorem[style=remark, name=Note, numbered=no]{note*}

% 使用 `\declaretheoremstyle` 命令定义新的 solutionstyle 样式,类似 proof 环境,但是引导词变成 Solution
\declaretheoremstyle[headfont=\bfseries, bodyfont=\normalfont, spaceabove=3pt, spacebelow=3pt, qed=\ensuremath{\square}]{solutionstyle}

% 采用新定义的 solutionstyle 样式,定义 `solution`/`solition*` 环境
\declaretheorem[style=solutionstyle, name=Solution, numbered=yes, numberwithin=section]{solution}
\declaretheorem[style=solutionstyle, name=Solution, numbered=no]{solution*}

% 导入 tcolorbox 宏包以使用盒子美化现有的定理环境
\usepackage[most]{tcolorbox}

% tcolorbox 宏包的功能非常复杂,这里只需要使用 `\tcolorboxenvironment` 命令
% 首先封装一个 `\newtcbenvironment` 命令
% 它可以同时为 `#1` 以及 `#1*` 这两个环境加上盒子,公共参数:
% - `#2`:在定义时传入的参数,这里主要是边框颜色和背景色
% - `enhanced`:样式增强
% - `breakable`:允许跨页
% - `boxrule=1pt`:边框宽度为 1pt
%
% 还有不同的参数:
% - `#1` 盒子使用直角边框(`sharp corners`)
% - `#1*` 盒子使用圆角边框(`rounded corners`)
%
% > 对 `\newtcbenvironment` 内部的公共参数部分进行调整,就可以实现所有盒子只保留左侧边框或者四周无边框等不同的效果。

\newcommand{\newtcbenvironment}[2]{
    \tcolorboxenvironment{#1}{#2, enhanced, breakable, sharp corners,leftrule=2pt, rightrule=0pt, toprule=0pt, bottomrule=0pt}
    \tcolorboxenvironment{#1*}{#2, enhanced, breakable, rounded corners,leftrule=2pt, rightrule=0pt, toprule=0pt, bottomrule=0pt}
}

% 下面就是为前面的各种定理环境加上盒子,参数是盒子的边框颜色 `colframe` 和背景色 `colback`
%
% 具体颜色如下表
%
% |            环境名             |   盒子边框颜色    |    盒子背景色    |
% | :---------------------------: | :---------------: | :--------------: |
% |   `theorem`, `law`    |    RoyalPurple    |  RoyalPurple!8   |
% | `corollary`, `lemma`, `claim` |     NavyBlue      |    SkyBlue!8     |
% |         `definition`          |    ForestGreen    |  ForestGreen!5   |
% |           `example`           |     RawSienna     |   RawSienna!5    |
% |           `problem`           | WildStrawberry!30 | WildStrawberry!5 |
% |          `solution`           |     Goldenrod     |  Goldenrod!10    |
%
% 说明:
%
% - 这里采用 `xcolor` 宏包所提供的标准颜色,`xx!n`代表将颜色 `xx` 以 `n%` 比例和白色混合得到的浅颜色。
% - 为了避免颜色过多,对语义类似的环境合并采用相同的盒子颜色。

% 定义 theorem (定理)环境
\newtcbenvironment{theorem}{colframe=RoyalPurple, colback=RoyalPurple!8}  
% 定义 law (定律)环境
\newtcbenvironment{law}{colframe=RoyalPurple, colback=RoyalPurple!8}
% 定义 corollary (推论)环境
\newtcbenvironment{corollary}{colframe=NavyBlue, colback=SkyBlue!8}
% 定义 lemma (引理)环境
\newtcbenvironment{lemma}{colframe=NavyBlue, colback=SkyBlue!8}
% 定义 claim (断言)环境
\newtcbenvironment{claim}{colframe=NavyBlue, colback=SkyBlue!8}

% 定义 definition (定义)环境
\newtcbenvironment{definition}{colframe=ForestGreen, colback=ForestGreen!5}
% 定义 example (例子)环境
\newtcbenvironment{example}{colframe=RawSienna, colback=RawSienna!5}
% 定义 problem (问题)环境
\newtcbenvironment{problem}{colframe=WildStrawberry!30, colback=WildStrawberry!5}
% 定义 proof (证明)环境
\newtcbenvironment{proof}{colframe=Goldenrod, colback=Goldenrod!10}

%===================================
% Part 和 Chapter 支持
%===================================

% \usepackage{titlesec}

% \providecommand{\partname}{Part}
% \providecommand{\chaptername}{Chapter}
% \providecommand{\partautorefname}{Part}
% \providecommand{\chapterautorefname}{Chapter}

% \titleclass{\chapter}{straight}[\section]
% \newcounter{chapter}
% \renewcommand{\thechapter}{\arabic{chapter}}

% \titleformat{\part}[display]
%   {\centering\normalfont\Huge\bfseries}
%   {\partname~\thepart}
%   {0pt}
%   {}
%   [\vspace{1ex}\titlerule]
% \titlespacing*{\part}{0pt}{-10pt}{30pt}

% \titleformat{\chapter}[display]
%   {\normalfont\Huge\bfseries}
%   {\chaptername~\thechapter}
%   {0pt}
%   {}
%   [\vspace{1ex}\titlerule]
% \titlespacing*{\chapter}{0pt}{-10pt}{24pt}

% \setcounter{secnumdepth}{3}
% \setcounter{tocdepth}{3}

% \renewcommand{\thesection}{\thechapter.\arabic{section}}
% \renewcommand{\thesubsection}{\thechapter.\arabic{section}.\arabic{subsection}}
% \renewcommand{\thesubsubsection}{\thechapter.\arabic{section}.\arabic{subsection}.\arabic{subsubsection}}

% \crefname{chapter}{Chapter}{Chapters}
% \Crefname{chapter}{Chapter}{Chapters}
% \crefname{part}{Part}{Parts}
% \Crefname{part}{Part}{Parts}

% \makeatletter
% \@addtoreset{section}{chapter}
% \@addtoreset{equation}{chapter}
% \renewcommand{\theequation}{\thechapter.\arabic{equation}}
% \renewcommand{\theHchapter}{\arabic{chapter}}
% \renewcommand{\theHsection}{\theHchapter.\arabic{section}}
% \renewcommand{\theHsubsection}{\theHsection.\arabic{subsection}}
% \renewcommand{\theHsubsubsection}{\theHsubsection.\arabic{subsubsection}}
% \renewcommand{\theHequation}{\theHchapter.\arabic{equation}}
% \providecommand{\@chapapp}{\chaptername}
% \providecommand{\l@chapter}{\@dottedtocline{0}{0em}{1.5em}}
% \makeatother

\usepackage{tikz-3dplot}
\usetikzlibrary{arrows.meta, decorations.markings, calc, backgrounds,positioning}
\usetikzlibrary{patterns, decorations.markings, arrows.meta, bending, calc, shadows}
\usetikzlibrary{arrows.meta, decorations.markings, , backgrounds, shapes.arrows, shadows}

% 定义颜色
\definecolor{myblue}{RGB}{30, 144, 255}
\definecolor{mygreen}{RGB}{50, 205, 50}
\definecolor{myyellow}{RGB}{255, 255, 224}
\definecolor{darkgray}{RGB}{40, 40, 40}

% --- 适配白色背景的配色方案 ---
\definecolor{boardbg}{HTML}{FFFFFF}     % 背景白
\definecolor{sphereoutline}{HTML}{000000} % 球轮廓:黑色
\definecolor{equator}{HTML}{DAA520}    % 赤道:暗金色/深黄
\definecolor{latpink}{HTML}{E9967A}    % 背景纬线:深肉色/淡红
\definecolor{areafill}{HTML}{D0F0C0}   % 区域填充:淡茶绿 (Tea Green)
\definecolor{gridgreen}{HTML}{2E8B57}  % 网格线:海绿色 (SeaGreen)
\definecolor{boundarygreen}{HTML}{006400} % 区域边界:深绿
\definecolor{pathblue}{HTML}{0000FF}   % 路径:纯蓝
\definecolor{textgreen}{HTML}{006400}  % 文本:深绿
\definecolor{pointyellow}{HTML}{FFD700}% 点颜色:金色

\usetikzlibrary{arrows.meta, decorations.markings, patterns, calc, backgrounds}
\usetikzlibrary{shapes.geometric, fit}
% --- 配色方案 (适配白底) ---
\definecolor{sphereoutline}{HTML}{000000} % 球轮廓: 黑
\definecolor{latpink}{HTML}{D87093}    % 纬线: 也是淡红/粉色
\definecolor{equator}{HTML}{DAA520}    % 赤道: 金色
\definecolor{pathblue}{HTML}{1E90FF}   % 路径: 亮蓝
\definecolor{labelgreen}{HTML}{228B22} % 标签: 森林绿
\definecolor{pointcolor}{HTML}{1E90FF} % 点: 蓝色