\section{3D Topological Insulator}

Let's consider the low-energy continuum model for prototypical 3D topological insulators such as $\mathrm{Bi}_2\mathrm{Se}_3$. The bulk electronic structure of $\mathrm{Bi}_2\mathrm{Se}_3$ near the Fermi level is described by two p-orbitals $P1_z^+$ and $P2_z^-$ with $\pm$ denoting parity. Defining the basis $\{ |P1_z^+, \uparrow \rangle, |P1_z^+, \downarrow \rangle, |P2_z^-, \uparrow \rangle, |P2_z^-, \downarrow \rangle \}$ and retaining the wave vector $\bm{k}$ up to quadratic order, the low-energy effective Hamiltonian around the $\Gamma$ point is given by:
\begin{equation}
    \begin{aligned}
    H_{\mathrm{eff}}(\bm{k}) &= 
    \begin{bmatrix} 
    M(\bm{k}) & 0 & A_1 k_z & A_2 k_- \\ 
    0 & M(\bm{k}) & A_2 k_+ & -A_1 k_z \\ 
    A_1 k_z & A_2 k_- & -M(\bm{k}) & 0 \\ 
    A_2 k_+ & -A_1 k_z & 0 & -M(\bm{k}) 
    \end{bmatrix}\\
    &= A_2 k_x \alpha_1 + A_2 k_y \alpha_2 + A_1 k_z \alpha_3 + M(\bm{k}) \alpha_4
\end{aligned}
\end{equation}
where $k_\pm = k_x \pm \mathrm{i} k_y$ and $M(\bm{k}) = m_0 - B_1 k_z^2 - B_2(k_x^2 + k_y^2)$. The $4 \times 4$ matrices $\alpha_\mu$ are given by the Dirac representation:
\begin{equation}
\alpha_j = \begin{bmatrix} 0 & \sigma_j \\ \sigma_j & 0 \end{bmatrix} , \quad \alpha_4 = \begin{bmatrix} 1 & 0 \\ 0 & -1 \end{bmatrix}
\end{equation}

Next, let's consider the lattice model of above Hamiltonian, which can calculate the $\mathbb{Z}_2$ invariant identifying whether a phase is topological nontrivial or trivial.

The simplest 3D lattice is the cubic lattice. We make the changes as following (ignore lattice constant $a$).
\begin{equation}
\left\{
\begin{aligned}
k_i &\to \sin k_i \\
k_i^2 &\to 2(1 - \cos k_i)
\end{aligned}
\right.
\end{equation}

and simplify the coefficients to obtain the isotropic lattice Hamiltonian:
\begin{equation}
H_{\mathrm{eff}}(\bm{k}) = A(\alpha_1 \sin k_x + \alpha_2 \sin k_y + \alpha_3 \sin k_z) + [m_0 + r \sum (1 - \cos k_i)] \alpha_4
\end{equation}
where we define $A=A_1=A_2$ and $r=-2B_1=-2B_2$.
Then, we use Fu-Kane method to calculate the $\mathbb{Z}_2$ topological invariant for this system:
\begin{equation}
(-1)^\nu = \prod_{i=1}^{N_{\mathrm{TRIM}}} \mathrm{sgn}[M(\Lambda_i)]
\end{equation}
In this 3D cubic lattice, there are 8 TRIM (Time-Reversal Invariant Momenta which are invariant under $k_i \to -k_i$) point :
\begin{equation}
\left\{
\begin{aligned}
&\Gamma : (0,0,0) \\
&X : (\uppi,0,0)/ (0,\uppi,0)/ (0,0,\uppi) \\
&M : (\uppi,\uppi,0)/ (\uppi,0,\uppi)/ (0,\uppi,\uppi) \\
&R : (\uppi,\uppi,\uppi)
\end{aligned}
\right.
\end{equation}

\begin{figure}[htbp]
    \centering
    \begin{tikzpicture}[
        % 设置3D投影视角,使得x轴指向左下,y轴指向右,z轴指向上
        % 这种设置对应于从第一卦限外观看立方体,使其一角朝向观察者
        x={(-0.6cm,-0.4cm)}, 
        y={(1cm,0cm)}, 
        z={(0cm,1cm)}, 
        scale=2,
        >=Stealth
    ]

        % --- 定义坐标点 ---
        % 晶格常数设为 1 (绘图范围 -1 到 1)
        \coordinate (O) at (0,0,0); % Gamma点
        
        % 高对称点 (基于简单立方Simple Cubic的布里渊区定义)
        % X: 面心 (沿着b2方向的面) -> (0, 1, 0)
        % R: 顶点 (最远的角) -> (1, 1, 1)
        % M: 棱心 (连接X和R的棱的中点?) 
        % 在标准SC中,M是(1,1,0)。在图中,M位于连接前脸和右脸的垂直棱的中点。
        % 几何上:X(0,1,0), M(1,1,0), R(1,1,1)
        
        \coordinate (Gamma) at (0,0,0);
        \coordinate (X) at (0,1,0);
        \coordinate (M) at (1,1,0);
        \coordinate (R) at (1,1,1);

        % 立方体顶点
        \coordinate (V1) at (-1,-1,-1);
        \coordinate (V2) at (1,-1,-1);
        \coordinate (V3) at (1,1,-1);
        \coordinate (V4) at (-1,1,-1);
        \coordinate (V5) at (-1,-1,1);
        \coordinate (V6) at (1,-1,1);
        \coordinate (V7) at (1,1,1); % R点位置
        \coordinate (V8) at (-1,1,1);

        % --- 1. 绘制背景虚线 (背面不可见的棱) ---
        % 背面顶点是 (-1,-1,-1), (-1,1,-1), (1,-1,-1) 等
        % 根据视角,面 x=-1, y=-1, z=-1 是背面的
        \draw[dotted, thick, darkgray] (V1) -- (V2);
        \draw[dotted, thick, darkgray] (V1) -- (V4);
        \draw[dotted, thick, darkgray] (V1) -- (V5);
        \draw[dotted, thick, darkgray] (V4) -- (V3); % 后上面
        \draw[dotted, thick, darkgray] (V2) -- (V3); % 后右面 (部分被遮挡,但在该视角下V3可见)
        % 修正:在该视角下(x左下, y右),(1,1,1)最靠近观察者。
        % (-1,-1,-1) 最远。与(-1,-1,-1)相连的棱都是虚线。
        
        % 为了确保透视正确,手动指定虚实
        % 只有连接到 (-1,-1,-1) 的三条棱是完全不可见的内部棱,或者是背面轮廓
        % 另外,由于立方体不透明,所有背面面上的线都应虚线。
        % 简单处理:画出完整的背面框架
        \draw[dotted, thick, darkgray] (-1,1,-1) -- (1,1,-1);
        \draw[dotted, thick, darkgray] (1,-1,-1) -- (1,1,-1);

        % --- 2. 绘制坐标轴 (b1, b2, b3) ---
        % b1 穿过 x=1 的面中心 (1,0,0)
        \draw[->, thick] (1,0,0) -- (1.6,0,0) node[anchor=east] {$\mathbf{b}_1$};
        % b2 穿过 y=1 的面中心 (0,1,0) (即X点)
        \draw[->, thick] (0,1,0) -- (0,1.6,0) node[anchor=west] {$\mathbf{b}_2$};
        % b3 穿过 z=1 的面中心 (0,0,1)
        \draw[->, thick] (0,0,1) -- (0,0,1.6) node[anchor=west] {$\mathbf{b}_3$};

        % --- 3. 绘制高对称路径 (红色四面体) ---
        % 内部连线 (Gamma点在体心),使用虚线
        \draw[red, very thick, dashed] (Gamma) -- (X);
        \draw[red, very thick, dashed] (Gamma) -- (M);
        \draw[red, very thick, dashed] (Gamma) -- (R);
        
        % 表面连线,保持实线
        \draw[red, very thick] (X) -- (M);
        \draw[red, very thick] (M) -- (R);
        \draw[red, very thick] (X) -- (R);

        % --- 4. 绘制可见的立方体外框 (实线) ---
        % 前面 (x=1), 右面 (y=1), 上面 (z=1)
        % 轮廓线
        \draw[black, thin] (-1,1,1) -- (1,1,1); % 上前棱
        \draw[black, thin] (1,1,1) -- (1,-1,1); % 前右棱 (垂直)
        \draw[black, thin] (1,1,1) -- (1,1,-1); % 右上棱
        
        \draw[black, thin] (-1,-1,1) -- (1,-1,1); % 下前
        \draw[black, thin] (-1,-1,1) -- (-1,1,1); % 左前
        
        \draw[black, thin] (1,-1,1) -- (1,-1,-1); % 右下
        \draw[black, thin] (-1,1,1) -- (-1,1,-1); % 左上
        
        % 补全外框闭合
        \draw[black, thin] (1,1,-1) -- (-1,1,-1);
        \draw[black, thin] (1,1,-1) -- (1,-1,-1);

        % 辅助虚线 (Gamma到轴的投影)
        % 指向 b1 方向 (右前方面中心)
        \draw[dotted, thick] (Gamma) -- (1,0,0);
        % 指向 b3 方向 (上方面中心)
        \draw[dotted, thick] (Gamma) -- (0,0,1);

        % --- 5. 绘制高对称点 (圆点) ---
        \fill[red] (Gamma) circle (1.5pt) node[anchor=north east, text=black] {$\Gamma$};
        \fill[red] (X) circle (1.5pt) node[anchor=south west, text=black] {X};
        \fill[red] (M) circle (1.5pt) node[anchor=north east, text=black] {M};
        \fill[red] (R) circle (1.5pt) node[anchor=south, text=black, yshift=2pt] {R};

    \end{tikzpicture}
    \caption{Highly symmetric points in the Brillouin zone of a cubic lattice}
    \label{fig:symmetric-points-of-cubic-lattice}
\end{figure}

The mass term is:
\begin{equation}
M(\bm{k}) = m_0 + r \sum_{t=x,y,z} (1 - \cos k_t)
\end{equation}
so we can calculate the mass term for each group of point:
\begin{equation}
\left\{
\begin{aligned}
&\Gamma : M = m_0 \\
&X : M = m_0 + 2r \\
&M : M = m_0 + 4r \\
&R : M = m_0 + 6r
\end{aligned}
\right.
\end{equation}
Now we substitute these mass terms into the $\mathbb{Z}_2$ topological invariant formula
\begin{equation}
\begin{aligned}
(-1)^\nu &= \mathrm{sgn}(m_0) \times \mathrm{sgn}(m_0 + 2r)^3 \times \mathrm{sgn}(m_0 + 4r)^3 \times \mathrm{sgn}(m_0 + 6r) \\
&= \mathrm{sgn}(m_0) \times \mathrm{sgn}(m_0 + 2r) \times \mathrm{sgn}(m_0 + 4r) \times \mathrm{sgn}(m_0 + 6r) \\
&= \begin{cases}
-1 & (0 > \frac{m_0}{r} > -2, \ -4 > \frac{m_0}{r} > -6) \\
1 & (\frac{m_0}{r} > 0, \ -2 > \frac{m_0}{r} > -4, \ -6 > \frac{m_0}{r})
\end{cases}
\end{aligned}
\end{equation}

let's return to the continuum isotropic Hamiltonian to obtain the $\theta$ term. Consider the Hamiltonian around the $\Gamma$ point can be simplified by ignoring the terms second-order in $k_i$:
\begin{equation}
H_{\mathrm{TI}}(\bm{k}) = A \bm{k} \cdot \bm{\alpha} + m_0 \alpha_4
\end{equation}