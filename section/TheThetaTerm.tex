\section{The theta term and Axion Electrodynamics}

In relativistic notation, the Maxwell action for electromagnetism takes a wonderfully compact form:
\begin{equation}
S_{Max} = \int \mathrm{d}^4x -\frac{1}{4} F^{\mu\nu} F_{\mu\nu} = \int \mathrm{d}^4x \frac{1}{2} (\bm{E}^2 - \bm{B}^2)
\end{equation}
Here:
\begin{equation}
\left\{
\begin{aligned}
    F_{\mu\nu} &= \partial_\mu A_\nu - \partial_\nu A_\mu \\
        & = \begin{pmatrix}
            0 & -E_x & -E_y & -E_z \\
            E_x & 0 & -B_z & B_y \\
            E_y & B_z & 0 & -B_x \\
            E_z & -B_y & B_x & 0
            \end{pmatrix}\\
    E_i &= F_{0i} \\
    F_{ij} &= -\epsilon_{ijk} B_k
\end{aligned}
\right.
\end{equation}
One reason that the Maxwell action is so simple is that there is very little else we can write down that is both gauge invariant and Lorentz invariant. 
There is, however one term that we can add to the Maxwell action is also both gauge invariant and Lorentz invariant, which we call it theta term:
\begin{equation}
S_\theta = \frac{\theta e^2}{4\mathrm{\pi}^2} \int \mathrm{d}^4x \frac{1}{4} {}^\star F^{\mu\nu} F_{\mu\nu} = -\frac{\theta e^2}{4\mathrm{\pi}^2} \int \mathrm{d}^4x \bm{E} \cdot \bm{B}
\end{equation}
which ${}^\star F^{\mu\nu}$ is the dual tensor:
\begin{equation}
    \begin{aligned}
    {}^\star F^{\mu\nu} &= \frac{1}{2} \epsilon^{\mu\nu\rho\sigma} F_{\rho\sigma} \\
    &= \begin{pmatrix}
        0 & -B_x & -B_y & -B_z \\
        B_x & 0 & E_z & -E_y \\
        B_y & -E_z & 0 & E_x \\
        B_z & E_y & -E_x & 0
        \end{pmatrix}
\end{aligned}
\end{equation}
and $\theta$ is a constant parameter.

However, the theta term is simple to check that it can be written as a total derivative:
\begin{equation}
\begin{aligned}
\int \mathrm{d}^4x \frac{1}{4} {}^\star F^{\mu\nu} F_{\mu\nu} &= \frac{1}{2} \int \mathrm{d}^4x \cdot \frac{1}{4} \epsilon^{\mu\nu\rho\sigma} F_{\rho\sigma} F_{\mu\nu} \\
&= \frac{1}{8} \int \mathrm{d}^4x \cdot \epsilon^{\mu\nu\rho\sigma} (\partial_\rho A_\sigma - \partial_\sigma A_\rho) (\partial_\mu A_\nu - \partial_\nu A_\mu) \\
&= \frac{1}{2} \int \mathrm{d}^4x \cdot \epsilon^{\mu\nu\rho\sigma} (\partial_\mu A_\nu \partial_\rho A_\sigma)
\end{aligned}
\end{equation}
Now let's proof a useful equation:
\begin{equation}
\epsilon^{\mu\nu\rho\sigma} A_\nu \partial_\mu \partial_\rho A_\sigma = 0
\end{equation}
Because:
\begin{equation}
\begin{aligned}
\epsilon^{\mu\nu\rho\sigma} A_\nu \partial_\mu \partial_\rho A_\sigma &= \epsilon^{\rho\nu\mu\sigma} A_\nu \partial_\rho \partial_\mu A_\sigma \\
&= \epsilon^{\rho\nu\mu\sigma} A_\nu \partial_\mu \partial_\rho A_\sigma \\
&= -\epsilon^{\mu\nu\rho\sigma} A_\nu \partial_\mu \partial_\rho A_\sigma
\end{aligned}
\end{equation}
So we can get:
\begin{equation}
2\epsilon^{\mu\nu\rho\sigma} A_\nu \partial_\mu \partial_\rho A_\sigma = 0 \Rightarrow \epsilon^{\mu\nu\rho\sigma} A_\nu \partial_\mu \partial_\rho A_\sigma = 0
\end{equation}

So the quadratic part of theta term become:
\begin{equation}
\begin{aligned}
\int \mathrm{d}^4x \frac{1}{4} {}^\star F^{\mu\nu} F_{\mu\nu} &= \frac{1}{2} \int \mathrm{d}^4x \epsilon^{\mu\nu\rho\sigma} (\partial_\mu A_\nu \partial_\rho A_\sigma + A_\nu \partial_\mu \partial_\rho A_\sigma) \\
&= \frac{1}{2} \int \mathrm{d}^4x \epsilon^{\mu\nu\rho\sigma} \partial_\mu (A_\nu \partial_\rho A_\sigma)
\end{aligned}
\end{equation}
The total derivative form of theta term is:
\begin{equation}
S_\theta = \frac{\theta e^2}{8\mathrm{\pi}^2} \int \mathrm{d}^4x \partial_\mu (\epsilon^{\mu\nu\rho\sigma} A_\nu \partial_\rho A_\sigma)
\end{equation}
We say that the theta term is topological. It depends only on boundary information. 
However, when deriving using the principle of least action to derive the field's equation of motion, 
the values of field are fixed on the infinite boundary. 
Therefore, the upshot is that the theta term does not change the equations of motion and, it would seem, can have little effect on the physics.

But under some situations that involve subtle interplay between quantum mechanics and topology, 
there are a number of interesting phenomena of physics which are led by theta term.

Also, we can look at the situations where $\theta$ affects the dynamics classically. This occurs when $\theta$ is not constant, but instead varies in space and time.

\begin{equation}
\theta = \theta(\bm{x}, t)
\end{equation}

So we can write down the axion electrodynamics action:
\begin{equation}
S = \int \mathrm{d}^4x (-\frac{1}{4} F^{\mu\nu} F_{\mu\nu} + \frac{e^2}{16\mathrm{\pi}^2} \theta(\bm{x}, t) {}^\star F^{\mu\nu} F_{\mu\nu})
\end{equation}

We use Euler-Lagrange equations:
\begin{equation}
\frac{\partial \mathcal{L}}{\partial A_\mu} - \partial_\nu (\frac{\partial \mathcal{L}}{\partial (\partial_\nu A_\mu)}) = 0
\end{equation}

to get the equations of axion electrodynamics

Because $\mathcal{L}$ doesn't depend on $A_\mu$, so the equations of motion are:
\begin{equation}
\frac{\partial \mathcal{L}}{\partial A_\mu} = 0 \Rightarrow \partial_\nu [\frac{\partial \mathcal{L}}{\partial (\partial_\nu A_\mu)}] = 0
\end{equation}

and the $\mathcal{L}$ of axion electrodynamics is $(\alpha = \frac{e^2}{4\mathrm{\pi}})$:
\begin{equation}
\begin{aligned}
\mathcal{L} &= \mathcal{L}_{Max} + \mathcal{L}_\theta \\
&= -\frac{1}{4} F^{\mu\nu} F_{\mu\nu} + \frac{\alpha}{4\mathrm{\pi}} \theta(\bm{x}, t) {}^\star F^{\mu\nu} F_{\mu\nu}
\end{aligned}
\end{equation}

First, let's see $\mathcal{L}_\theta$:
\begin{equation}
\begin{aligned}
\frac{\partial \mathcal{L}}{\partial (\partial_\nu A_\mu)} &= \frac{\partial}{\partial (\partial_\nu A_\mu)} \cdot \frac{\alpha}{4\mathrm{\pi}} \theta(\bm{x}, t) \cdot \frac{1}{2} \epsilon^{\alpha\beta\rho\sigma} F_{\alpha\beta} F_{\rho\sigma} \\
&= \frac{\alpha}{8\mathrm{\pi}} \{ [\frac{\partial}{\partial (\partial_\nu A_\mu)} \theta(\bm{x}, t)] \cdot \epsilon^{\alpha\beta\rho\sigma} F_{\alpha\beta} F_{\rho\sigma} + \theta(\bm{x}, t) [\frac{\partial \epsilon^{\alpha\beta\rho\sigma} F_{\alpha\beta} F_{\rho\sigma}}{\partial (\partial_\nu A_\mu)}] \}
\end{aligned}
\end{equation}

let's see the first term:
\begin{equation}
\epsilon^{\alpha\beta\rho\sigma} F_{\alpha\beta} F_{\rho\sigma} \frac{\partial}{\partial (\partial_\nu A_\mu)} \theta(\bm{x}, t) = 0
\end{equation}

then let's see the second term:
\begin{equation}
\begin{aligned}
\theta(\bm{x}, t) \cdot \frac{\partial}{\partial (\partial_\nu A_\mu)} \cdot \epsilon^{\alpha\beta\rho\sigma} F_{\alpha\beta} F_{\rho\sigma} &= 4 \theta(\bm{x}, t) (\frac{\partial}{\partial (\partial_\nu A_\mu)} \partial_\alpha A_\beta \partial_\rho A_\sigma) \epsilon^{\alpha\beta\rho\sigma} \\
&= 4 \theta(\bm{x}, t) (\delta^\nu_\alpha \delta^\mu_\beta \partial_\rho A_\sigma + \delta^\nu_\rho \delta^\mu_\sigma \partial_\alpha A_\beta) \cdot \epsilon^{\alpha\beta\rho\sigma} \\
&= 4 \theta(\bm{x}, t) \cdot \epsilon^{\nu\mu\rho\sigma} (\partial_\rho A_\sigma - \partial_\sigma A_\rho) \\
&= 8 \theta(\bm{x}, t) {}^\star F^{\nu\mu}
\end{aligned}
\end{equation}

next, we can get:
\begin{equation}
\partial_\nu [\theta(\bm{x}, t) {}^\star F^{\nu\mu}] = {}^\star F^{\nu\mu} \partial_\nu \theta(\bm{x}, t) + \theta(\bm{x}, t) \cdot \partial_\nu {}^\star F^{\nu\mu}
\end{equation}

using the Bianchi identities:
\begin{equation}
\partial_\mu {}^\star F^{\mu\nu} = 0
\end{equation}

so we can get:
\begin{equation}
\partial_\nu \theta(\bm{x}, t) {}^\star F^{\nu\mu} = {}^\star F^{\nu\mu} \partial_\nu \theta(\bm{x}, t)
\end{equation}

Finally, we can get:
\begin{equation}
\partial_\nu [\frac{\partial \mathcal{L}_\theta}{\partial (\partial_\nu A_\mu)}] = \frac{\alpha}{\mathrm{\pi}} {}^\star F^{\nu\mu} \cdot \partial_\nu \theta(\bm{x}, t)
\end{equation}

And the Lagrangian of Maxwell part:
\begin{equation}
\partial_\nu [\frac{\partial \mathcal{L}_{Max}}{\partial (\partial_\nu A_\mu)}] = - \partial_\nu F^{\nu\mu}
\end{equation}

So the equations of axion electrodynamics is:
\begin{equation}
\left\{
\begin{aligned}
\partial_\nu F^{\nu\mu} &= \frac{\alpha}{\mathrm{\pi}} {}^\star F^{\nu\mu} \cdot \partial_\nu \theta(\bm{x}, t) \\
\partial_\mu {}^\star F^{\mu\nu} &= 0
\end{aligned}
\right.
\end{equation}

In the end, the deformed Maxwell equations which are the equations of axion electrodynamics are:
\begin{equation}
\left\{
\begin{aligned}
\nabla \cdot \bm{E} &= \frac{\alpha}{\mathrm{\pi}} (\nabla \theta) \cdot \bm{B} \\
\nabla \times \bm{B} &= \frac{\partial \bm{E}}{\partial t} - \frac{\alpha}{\mathrm{\pi}} (\dot{\theta} \bm{B} + \nabla \theta \times \bm{E}) \\
\nabla \cdot \bm{B} &= 0 \\
\nabla \times \bm{E} &= - \frac{\partial \bm{B}}{\partial t}
\end{aligned}
\right.
\end{equation}

The first equation tells us that in regions of space where $\theta$ varies, a magnetic field $\bm{B}$ acts like an electric charge density. The second equation tells us that the combination $(\dot{\theta}\bm{B} + \nabla\theta \times \bm{E})$ acts like a current density

\subsection{The periodicity of theta and its values under Parity or Time - Reversal Symmetry}

In quantum theory, $\theta$ is a periodic variable: it lies in the range:
\begin{equation}
\theta \in [0, 2\mathrm{\pi})
\end{equation}
After imposing appropriate boundary conditions, $S_\theta$ can only take values of the form:
\begin{equation}
S_\theta = \theta N \quad \text{with } N \in \mathbb{Z}
\end{equation}
This means that the theta angle contributes to the partition function as
\begin{equation}
e^{\mathrm{i}S_\theta} = e^{\mathrm{i}N\theta}
\end{equation}
To show that $S_\theta$ must take the form, we consider a compact Euclidean spacetime which we take to $T^4$ and we take each of the circles in the torus to have radii $R$.

We consider easy case:
\begin{equation}
\left\{
\begin{aligned}
\bm{E} &= (0, 0, E_z), \ E_z = F_{03} = \partial_0 A_3 - \partial_3 A_0 \\
\bm{B} &= (0, 0, B_z), \ B_z = F_{21} = \partial_2 A_1 - \partial_1 A_2
\end{aligned}
\right.
\end{equation}
The integral of $S_\theta$ is:
\begin{equation}
I = \int_{T^4} \mathrm{d}^4x E_z \cdot B_z = \int_{T^4} \mathrm{d}x^0 \mathrm{d}x^3 \cdot E_z \cdot \int_{T^4} \mathrm{d}x^1 \mathrm{d}x^2 \cdot B_z
\end{equation}
The gauge field $A_\mu$ must be well defined on the underlying torus, which will put restrictions on the allowed values of $E_z$ and $B_z$. So the integral can't take any value.

Now let consider the restrictions of $A_\mu$ if $A_\mu$ be defined on torus. When a direction of space, say $x^1$, is periodic with radius $R$, the physics state in original point ($x^1=0$) must be identical to the final physics state:
\begin{equation}
\text{state}[A_1(0)] = \text{state}[A_1(0)]
\end{equation}