\begin{figure}[htbp]
    \centering
    \begin{tikzpicture}[
        % 设置3D投影视角,使得x轴指向左下,y轴指向右,z轴指向上
        % 这种设置对应于从第一卦限外观看立方体,使其一角朝向观察者
        x={(-0.6cm,-0.4cm)}, 
        y={(1cm,0cm)}, 
        z={(0cm,1cm)}, 
        scale=2,
        >=Stealth
    ]

        % --- 定义坐标点 ---
        % 晶格常数设为 1 (绘图范围 -1 到 1)
        \coordinate (O) at (0,0,0); % Gamma点
        
        % 高对称点 (基于简单立方Simple Cubic的布里渊区定义)
        % X: 面心 (沿着b2方向的面) -> (0, 1, 0)
        % R: 顶点 (最远的角) -> (1, 1, 1)
        % M: 棱心 (连接X和R的棱的中点?) 
        % 在标准SC中,M是(1,1,0)。在图中,M位于连接前脸和右脸的垂直棱的中点。
        % 几何上:X(0,1,0), M(1,1,0), R(1,1,1)
        
        \coordinate (Gamma) at (0,0,0);
        \coordinate (X) at (0,1,0);
        \coordinate (M) at (1,1,0);
        \coordinate (R) at (1,1,1);

        % 立方体顶点
        \coordinate (V1) at (-1,-1,-1);
        \coordinate (V2) at (1,-1,-1);
        \coordinate (V3) at (1,1,-1);
        \coordinate (V4) at (-1,1,-1);
        \coordinate (V5) at (-1,-1,1);
        \coordinate (V6) at (1,-1,1);
        \coordinate (V7) at (1,1,1); % R点位置
        \coordinate (V8) at (-1,1,1);

        % --- 1. 绘制背景虚线 (背面不可见的棱) ---
        % 背面顶点是 (-1,-1,-1), (-1,1,-1), (1,-1,-1) 等
        % 根据视角,面 x=-1, y=-1, z=-1 是背面的
        \draw[dotted, thick, darkgray] (V1) -- (V2);
        \draw[dotted, thick, darkgray] (V1) -- (V4);
        \draw[dotted, thick, darkgray] (V1) -- (V5);
        \draw[dotted, thick, darkgray] (V4) -- (V3); % 后上面
        \draw[dotted, thick, darkgray] (V2) -- (V3); % 后右面 (部分被遮挡,但在该视角下V3可见)
        % 修正:在该视角下(x左下, y右),(1,1,1)最靠近观察者。
        % (-1,-1,-1) 最远。与(-1,-1,-1)相连的棱都是虚线。
        
        % 为了确保透视正确,手动指定虚实
        % 只有连接到 (-1,-1,-1) 的三条棱是完全不可见的内部棱,或者是背面轮廓
        % 另外,由于立方体不透明,所有背面面上的线都应虚线。
        % 简单处理:画出完整的背面框架
        \draw[dotted, thick, darkgray] (-1,1,-1) -- (1,1,-1);
        \draw[dotted, thick, darkgray] (1,-1,-1) -- (1,1,-1);

        % --- 2. 绘制坐标轴 (b1, b2, b3) ---
        % b1 穿过 x=1 的面中心 (1,0,0)
        \draw[->, thick] (1,0,0) -- (1.6,0,0) node[anchor=east] {$\mathbf{b}_1$};
        % b2 穿过 y=1 的面中心 (0,1,0) (即X点)
        \draw[->, thick] (0,1,0) -- (0,1.6,0) node[anchor=west] {$\mathbf{b}_2$};
        % b3 穿过 z=1 的面中心 (0,0,1)
        \draw[->, thick] (0,0,1) -- (0,0,1.6) node[anchor=west] {$\mathbf{b}_3$};

        % --- 3. 绘制高对称路径 (红色四面体) ---
        % 内部连线 (Gamma点在体心),使用虚线
        \draw[red, very thick, dashed] (Gamma) -- (X);
        \draw[red, very thick, dashed] (Gamma) -- (M);
        \draw[red, very thick, dashed] (Gamma) -- (R);
        
        % 表面连线,保持实线
        \draw[red, very thick] (X) -- (M);
        \draw[red, very thick] (M) -- (R);
        \draw[red, very thick] (X) -- (R);

        % --- 4. 绘制可见的立方体外框 (实线) ---
        % 前面 (x=1), 右面 (y=1), 上面 (z=1)
        % 轮廓线
        \draw[black, thin] (-1,1,1) -- (1,1,1); % 上前棱
        \draw[black, thin] (1,1,1) -- (1,-1,1); % 前右棱 (垂直)
        \draw[black, thin] (1,1,1) -- (1,1,-1); % 右上棱
        
        \draw[black, thin] (-1,-1,1) -- (1,-1,1); % 下前
        \draw[black, thin] (-1,-1,1) -- (-1,1,1); % 左前
        
        \draw[black, thin] (1,-1,1) -- (1,-1,-1); % 右下
        \draw[black, thin] (-1,1,1) -- (-1,1,-1); % 左上
        
        % 补全外框闭合
        \draw[black, thin] (1,1,-1) -- (-1,1,-1);
        \draw[black, thin] (1,1,-1) -- (1,-1,-1);

        % 辅助虚线 (Gamma到轴的投影)
        % 指向 b1 方向 (右前方面中心)
        \draw[dotted, thick] (Gamma) -- (1,0,0);
        % 指向 b3 方向 (上方面中心)
        \draw[dotted, thick] (Gamma) -- (0,0,1);

        % --- 5. 绘制高对称点 (圆点) ---
        \fill[red] (Gamma) circle (1.5pt) node[anchor=north east, text=black] {$\Gamma$};
        \fill[red] (X) circle (1.5pt) node[anchor=south west, text=black] {X};
        \fill[red] (M) circle (1.5pt) node[anchor=north east, text=black] {M};
        \fill[red] (R) circle (1.5pt) node[anchor=south, text=black, yshift=2pt] {R};

    \end{tikzpicture}
    \caption{Highly symmetric points in the Brillouin zone of a cubic lattice}
    \label{fig:symmetric-points-of-cubic-lattice}
\end{figure}