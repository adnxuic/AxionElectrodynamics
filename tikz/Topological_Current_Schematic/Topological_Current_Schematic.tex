\begin{figure}[htbp]
    \centering
    \begin{tikzpicture}[
        x={(-0.5cm,-0.3cm)}, y={(1cm,0cm)}, z={(0cm,1cm)}, % 3D坐标系
        scale=2,
        >=Stealth,
        font=\sffamily
    ]

        % --- 参数设置 ---
        \def\w{3.0} % 宽度 (y轴)
        \def\d{2.0} % 深度 (x轴)
        \def\h{1.2} % 高度 (z轴)
        
        % 颜色定义
        \definecolor{materialcolor}{RGB}{230, 230, 235} % 浅灰材料色
        \definecolor{efieldcolor}{RGB}{220, 50, 50}     % 电场红
        \definecolor{currentcolor}{RGB}{255, 140, 0}    % 电流橙色

        % 定义中间箭头的样式
        \tikzset{->-/.style={decoration={
        markings,
        mark=at position #1 with {\arrow{>}}},postaction={decorate}}}

        % --- 1. 背景电场 (Background E Field) ---
        % 绘制在材料后方/下方的电场线,衬托整体环境
        \foreach \iz in {0.2, 0.6, 1.0} {
            % 在 x=0 平面后方一点
            \draw[efieldcolor, thin, opacity=0.3] (0, -1.0, \iz) -- (0, 0, \iz);
        }
        % 在材料左侧(y<0)的延伸
        \foreach \ix in {0.5, 1.5} {
            \foreach \iz in {0.4, 0.8} {
                \draw[efieldcolor, thin, opacity=0.3] (\ix, -1.0, \iz) -- (\ix, 0, \iz);
            }
        }

        % --- 2. 绘制立方体后/底/左面 (被遮挡面) ---
        % 顶点定义
        \coordinate (O) at (0,0,0);
        \coordinate (A) at (\d,0,0);
        \coordinate (B) at (\d,\w,0);
        \coordinate (C) at (0,\w,0);
        \coordinate (D) at (0,0,\h);
        \coordinate (E) at (\d,0,\h);
        \coordinate (F) at (\d,\w,\h);
        \coordinate (G) at (0,\w,\h);

        % 后棱 (虚线)
        \draw[dashed, gray!60] (O) -- (C);
        \draw[dashed, gray!60] (C) -- (G);
        \draw[dashed, gray!60] (C) -- (B);

        % 填充底层面 (Bottom & Left)
        \fill[materialcolor, opacity=0.8] (O) -- (A) -- (B) -- (C) -- cycle; % 底面
        \fill[materialcolor, opacity=0.6] (O) -- (A) -- (E) -- (D) -- cycle; % 左侧面

        % --- 3. 内部电场 (Inside E Field) ---
        % **关键步骤**:在画前表面之前画内部线,产生“被包裹”的透视感
        \foreach \ix in {0.5, 1.0, 1.5} { % 沿x轴分布
            \foreach \iz in {0.4, 0.8} {   % 沿z轴分布
                % 绘制贯穿内部的线,改为 dashed (虚线)
                \draw[efieldcolor, thick, dashed, opacity=0.8] (\ix, 0, \iz) -- (\ix, \w, \iz);
                % 绘制穿出体外的部分,改为实线 (solid) 并带箭头
                \draw[efieldcolor, thick, ->] (\ix, \w, \iz) -- (\ix, \w+1.0, \iz);
            }
        }

        % --- 4. 绘制立方体前/顶面 (Front & Top) ---
        % 前侧面 (Front Face) - 设为半透明,让步骤3的线透出来
        \fill[materialcolor, opacity=0.4] (A) -- (B) -- (F) -- (E) -- cycle; 
        
        % 顶面 (Top Face) - 不透明度稍高,承载表面电流
        \fill[materialcolor, opacity=0.9] (D) -- (E) -- (F) -- (G) -- cycle; 

        % 绘制边框线条
        \draw[thick, gray!80] (O) -- (A) -- (B); 
        \draw[thick, gray!80] (O) -- (D); 
        \draw[thick, gray!80] (A) -- (E); 
        \draw[thick, gray!80] (B) -- (F); 
        \draw[thick, gray!80] (D) -- (E) -- (F) -- (G) -- cycle; 

        % --- 5. 表面效应 (Surface Current) ---
        \foreach \iy in {0.5, 1.5, 2.5} {
            \draw[currentcolor, ultra thick, ->] (0.2, \iy, \h) -- (\d-0.2, \iy, \h);
        }
        
        % --- 6. 顶部/外部电场 (External E Field) ---
        % 在材料上方绘制多条电场线
        \def\zoffset{\h + 0.6}
        \foreach \ix in {0.5, 1.5} {
            % 第一层上方
            \draw[efieldcolor, thick, ->] (\ix, -0.8, \zoffset) -- (\ix, \w+0.5, \zoffset);
            % 第二层上方
            \draw[efieldcolor, thick, ->] (\ix, -0.8, \zoffset + 0.6) -- (\ix, \w+0.5, \zoffset + 0.6);
        }
        \node[efieldcolor, right] at (1.5, \w+0.5, \zoffset) {$\mathbf{E}$};

        % --- 7. 标注与文字 ---
        \node[anchor=south west, gray!60, scale=0.8] at (0,0,0) {Topological Material};
        \node at (\d/2, \w/2, \h+0.85) {Vacuum ($\theta = 0$)};
        
        % θ=π 标签
        \node[scale=0.9] at (\d/2, \w/2, \h/2 + 0.2) {\large $\theta = \pi$};

        % --- Surface Current 标签修改 ---
        % 移到左侧 (y < 0)
        % 从靠近左边缘的一条电流线 (y=0.5) 引出,向左延伸
        % 修改:将引线的终点和标签的位置从 y=-1.5 改为 y=-0.5,使其更靠近箭头
        % \draw[currentcolor, thin] (1.0, 0.5, \h) -- (1.0,0, \h); 
        \node[anchor=east, align=right, scale=0.9] at (1.0,0, \h) {
            \textcolor{currentcolor}{\textbf{Surface Current}} \\
            $K = \frac{\alpha c}{\pi} (\partial_z \theta) E_y$
        };

        % 坐标轴指示
        \begin{scope}[shift={(3.5, 0, 0)}] 
            \draw[->, thick, black!70] (0, 0, 0) -- (0, 0.6, 0) node[right, scale=0.8] {$y (\mathbf{E})$};
            \draw[->, thick, black!70] (0, 0, 0) -- (0, 0, 0.6) node[above, scale=0.8] {$z$};
            \draw[->, thick, black!70] (0, 0, 0) -- (0.6, 0, 0) node[below left, scale=0.8] {$x (\mathbf{K})$};
        \end{scope}

    \end{tikzpicture}
    \caption{Topological Current Schematic}
    \label{fig:topological_current_schematic}
\end{figure}