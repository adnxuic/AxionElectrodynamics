\begin{figure}[htbp]
    \centering
    \begin{tikzpicture}
        % --- 定义半径 ---
        \def\rInner{2.0} % 内圆半径
        \def\rOuter{4.0} % 外圆半径
        \def\dotSize{0.25} % 中心点半径
        \def\arrowOffset{0.15} % 红色箭头相对于蓝色箭头的偏移量

        % --- 绘制区域 ---
        % 1. 外圈 (灰色背景)
        \filldraw[fill=gray!10, draw=black] (0,0) circle (\rOuter);
        
        % 2. 内圈 (白色背景)
        \filldraw[fill=white, draw=black] (0,0) circle (\rInner);

        % --- 绘制边界上的正电荷 (新增部分) ---
        % 在内圆圆周上均匀分布 "+" 符号
        % 向外移动一点 (rInner + 0.25),去掉 fill=white
        \foreach \angle in {0, 22.5, ..., 337.5} {
            \node[red, font=\small\bfseries, inner sep=0.5pt] at (\angle:\rInner + 0.25) {+};
        }

        % --- 绘制中心粒子 ---
        \fill[blue] (0,0) circle (\dotSize);

        % --- 绘制矢量场 (循环8个方向) ---
        \foreach \angle in {0, 45, ..., 315} {
            
            % --- 蓝色矢量 (B场) ---
            % 从中心点边缘出发,延伸到接近外圆边缘
            \draw[blue, -{Stealth[length=3mm, width=2mm]}] 
                (\angle:\dotSize) -- (\angle:3.5);

            % --- 红色矢量 (E场) ---
            % 只在外圈存在,从内圆边界出发
            % 偏移逻辑:($ coordinate + (angle-90:offset) $) 
            % 这样可以确保红色箭头始终位于蓝色箭头的"右侧" (顺时针侧)
            \draw[red, -{Stealth[length=3mm, width=2mm]}] 
                ($(\angle:\rInner) + (\angle-90:\arrowOffset)$) -- 
                ($(\angle:3.5) + (\angle-90:\arrowOffset)$);
        }

        % --- 添加文字标签 ---
        
        % 区域标签
        % 向右移动避开正上方的箭头,放在约 70 度方向
        \node at (75:3.3) [fill=gray!10, inner sep=2pt] {\large $\theta \neq 0$};
        \node at (45:1.0) {\large $\theta=0$};

        % 场标签 (B 和 E) - 放在顶部的箭头旁边
        \node[blue] at (95:2.5) {\Large B};
        \node[red]  at (85:2.5) {\Large E};

    \end{tikzpicture}
    \caption{Witten Effect}
    \label{fig: Witten Effect}
\end{figure}